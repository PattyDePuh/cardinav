\documentclass[a4paper,11pt]{article}
\usepackage[ngerman]{babel}
\usepackage[ansinew]{inputenc}
\usepackage[round]{natbib}

\begin{document}
\title{Ein Test}
\author{Jan-Henrik Haunert}
\maketitle

Dies ist ein Test, so wie er auch schon von \citet{Gemsa2011} vorbildlich vollzogen wurde.
Oft werden Tests dagegen ohne den n�tigen Sachverstand durchgef�hrt \citep{fhssw-alfr-12}.

\begin{thebibliography}{2}
\providecommand{\natexlab}[1]{#1}
\providecommand{\url}[1]{\texttt{#1}}
\expandafter\ifx\csname urlstyle\endcsname\relax
  \providecommand{\doi}[1]{doi: #1}\else
  \providecommand{\doi}{doi: \begingroup \urlstyle{rm}\Url}\fi

\bibitem[Fink et~al.(2012)Fink, Haunert, Schulz, Spoerhase, und
  Wolff]{fhssw-alfr-12}
M.~Fink, J.-H. Haunert, A.~Schulz, J.~Spoerhase, und A.~Wolff.
\newblock Algorithms for labeling focus regions.
\newblock \emph{IEEE Transactions on Visualization and Computer Graphics},
  18\penalty0 (12):\penalty0 2583--2592, 2012.

\bibitem[Gemsa et~al.(2011)Gemsa, Haunert, und N\"{o}llenburg]{Gemsa2011}
A.~Gemsa, J.-H. Haunert, and M.~N\"{o}llenburg.
\newblock Boundary-labeling algorithms for panorama images.
\newblock In \emph{Proc. 19th ACM SIGSPATIAL International Conference on
  Advances in Geographic Information Systems (ACM GIS'11)}, pages 289--298,
  2011.

\end{thebibliography}

\end{document}